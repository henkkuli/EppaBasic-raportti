% Täällä kerrotaan sivuston sekä kääntäjän teknisestä toteutuksesta.

\section{Tekninen toteutus}
Jotta EppaBasic toimisi mahdollisimman monella
tietokonejärjestelmällä laitteistosta ja
käyttöjärjestelmästä riippumatta,
päätimme toteuttaa kaiken ohjelmointi-
ja suoritusympäristöä myöten
selainpohjaisesti.
Näin EppaBasic toimii niin
Windows-, Linux- ja OS X -koneilla
kuin myös monilla mobiililaitteilla.
Lisäksi modernit selaimet tarjoavat
monia hyödyllisiä ominaisuuksia
kuten yksinkertaisen tavan toteuttaa
käyttöliittymiä (HTML \cite{w3c_html})
sekä tehokkaan tavan piirtää grafiikkaa
näytölle (HTML Canvas \cite{mdn_canvas}).
Koko ohjelmakoodi EppaBasicin takana
on kirjoitettu JavaScriptillä,
joka on ainoa yleisesti selaimissa
tuettu ohjelmointikieli \cite{mdn_about_js}.

JavaScript-koodi EppaBasicin takana
on jaettu moduuleihin koodin
rakenteen selkeyttämiseksi.
Moduulien lataamiseen käytettään
RequireJS-kirjastoa \cite{requirejs}.
Moduulien käyttämällä voi kirjoittaa
selkeämpää koodia kuin ilman moduuleja,
koska moduulien riippuvuudet ovat
näkyvissä moduulin määrittelyssä.
RequireJS mahdollistaa myös kooditiedostojen
yhdistämisen ja pienentämisen ennen selaimelle
lähettämistä käyttäen r.js-ohjelmaa \cite{r.js}.

EppaBasicin kääntäjä kääntää ohjelmat
asm.js-koodiksi \cite{asm.js}.
Asm.js-koodi on tavallista
JavaScript-koodia, jonka
ominaisuuksia on rajoitettu siten,
että jotkin selaimet osaavat
suorittaa koodia tavallista
nopeammin.
Esimerkiksi uudet Mozilla Firefox
ja Google Chrome versiot
suorittavat
asm.js-koodia nopeammin
kuin vastaavaa tavallista
JavaScript-koodia.
Asm.js-koodin käyttäminen
ei kuitenkaan rajoita
EppaBasicin ominaisuuksia,
sillä asm.js-koodista
voidaan kutsua myös
tavallisia JavaScript-funktioita,
joissa voidaan toteuttaa
puuttuvat toiminnallisuudet.


\subsection{Kääntäjä}
Eräs tapa hahmottaa kääntäjää
on jakaa se kolmeen osaan:
ensivaiheeseen (eng. front-end),
välivaiheeseen (eng. middleware) ja
loppuvaiheeseen (eng. back-end).
EppaBasicin kääntäjä koostuu vain
ensivaiheesta ja loppuvaiheesta,
sillä EppaBasic ei vielä toistaiseksi
tarvitse välivaihetta.
Tulevaisuudessa välivaiheeseen
on tarkoitus toteuttaa muun muassa
optimoija, joka nopeuttaa
koodin suoritusta.

\subsubsection{Ensivaihe}
Ensivaihe on kääntämisen ensimmäinen vaihe.
Sen tehtävä on tulkita käyttäjän kirjoittama
EppaBasic-koodi ja muuttaa se kääntäjän
sisäiseen muotoon.
EppaBasicissa ensivaihe on myös ainoa
vaihe, joka voi kertoa käyttäjälle
koodissa olevista virheistä.

EppaBasicin kääntäjä käyttää sisäisenä
esitysmuotonaan abstraktia syntaksipuuta
(eng. abstract syntax tree, AST).
Siinä puun juuri vastaa koko ohjelmaa,
ja jokainen solmu vastaa tiettyä
syntaksillista osaa.
Esimerkiksi komentorakenteilla,
laskuoperaattoreilla ja
funktiomäärityksillä
on omat solmunsa, joilla on eri määrä lapsia.
Puun lehtiä ovat esimerkiksi
muuttujaviittaukset ja vakiot.

EppaBasicin ensivaihe muodostuu
kolmesa osasta:
selaajasta (eng. lexer),
jäsentäjästä (eng. parser) ja
tyyppientarkastajasta.

Selaaja lukee lähdekoodin alusta loppuun
ja jakaa sen merkityksellisiin osiin,
alkionimiin (eng. token),
Alkionimiä ovat erilaiset kielen
avainsanat (\eb{If}, \eb{Then}, \eb{EndIf}, jne.),
luvut (\eb{15}, \eb{-30}, jne.),
merkkijonot (\eb{"HelloWorld"}, \eb{"Tekstiä"}, jne.),
muuttujien ja funktioiden nimet (\eb{Print}, \eb{x}, jne.),
kommentit (\eb{' Kommentit alkavat heittomerkillä}, jne.),
erikoismerkit (\eb{+}, \eb{-}, \eb{^}, jne.) sekä
rivinvaihdot ja välilyöntimerkit.
Jos selaaja löytää pätkän koodia,
jota se ei voi jakaa kuuluvaksi
edellä mainittuihin ryhmiin,
selaaja kertoo käyttäjälle virheestä.

Jäsentäjä tulkitsee selaajalta saamansa
alkionimivirran ja muodostaa sen pohjalta
ohjelmakoodin AST-puun.
EppaBasicin jäsentäjä on tyypiltään
\emph{käsinkirjoitettu, rekursiivinen,
ylhäältä-alas-jäsentäjä}
(eng. \emph{hand-coded, recursive-descent,
top-down parser} \cite[luku 3.3.2]{eac2e}).
EppaBasic käyttää tällaista jäsentäjää,
koska se on yksinkertainen toteuttaa,
helposti ymmärrettävä ja tehokas.
Näin ominaisuuksien lisääminen
kääntäjään on helppoa.

Rekursiivinen ylhäältä-alas-jäsentäjä
toimii siten, että funktion aluksi
se lukee seuraavan alkionimen virrasta,
minkä perusteella se kutsuu seuraavaa
funktiota.
Funktiot palauttavat muodostamansa
AST-alipuun.

Jäsentämisen jälkeen tyyppientarkastaja
käy rekursiivisesti läpi jäsentäjän
luoman AST-puun.
Se pitää kirjaa jokaisessa kohdassa
näkyvistä muuttujista ja niiden tyypeistä.
Näin tyyppientarkastaja voi päätellä
jokaiselle lausekkeelle tyypin
ja kutsuttavien funktioiden
paluuarvojen tyypit.
Samalla tyyppientarkastaja selvittää,
mihin määritykseen jokainen muuttujan
käyttö viittaa ja mihin funktiomääritykseen
jokainen funktiokutus viittaa.

\subsubsection{Loppuvaihe}
Loppuvaihe on kääntämisen viimeinen vaihe,
joka muodostaa sisäisestä esitysmuodosta
suoritettavan ohjelman.
EppaBasicin loppuvaihe muuttaa ohjelman
asm.js-yhteensopi\-vaksi \cite{asm.js}
JavaScript-koodiksi.

EppaBasicin loppuvaihe käy rekursiivisesti
läpi AST-puun.
Jokainen AST-puun solmu muutetaan
vastaavaksi pätkäksi asm.js-koodia.

EppaBasicin loppuvaihetta monimutkaistaa se,
että EppaBasic mahdollistaa ruudun piirtämisen
keskellä komentorakenteita.
JavaScriptissä (ja siten myös asm.js:ssä)
ruudun päivittäminen ja käyttäjän syötteiden
lukeminen vaatii sen, että JavaScript-koodin
suorittaminen loppuu hetkeksi.
JavaScriptissä ei kuitenkaan ole
mahdollista tallentaa suorituksen tilaa
komentorakenteessa ja palata siihen.
Tämän takia kääntäjän loppuvaihe
joutuu muokkaamaan käännettävää
koodia siten, että suoritusta
voidaan näennäisesti jatkaa
keskeltä komentorakennetta.

Loppuvaihe merkitsee jokaiselle
AST-puun solmulle atomisuuden
eli jakamattomuuden.
Jakamattomuus merkitsee sitä,
että suoritusta ei tarvitse eikä voi
katkaista kyseisen solmun sisällä.
Kaikki solmut ovat aluksi atomisia.
Sen jälkeen merkitään jokainen
funktion \eb{DrawScreen} (päivittää ruudun)
kutsu epäatomiseksi.
Tämän jälkeen merkitään solmuja
epäatomisiksi, jos jokin niiden
lapsista on epäatominen tai
kutsuttava funktio on epäatominen.
Näin epäatomisuus peritytyy
puussa ylöspäin.

Nyt loppuvaihe voi jakaa koodin
mahdollisimman pitkiin atomisiin osiin.
Vaihe luo jokaiselle atomiselle osalle oman
funktionsa ja antaa funktioille indeksin.
Ohjelman suorituksen aluksi kutsupinossa
on yksi alkio: ohjelman ensimmäinen
atominen osa.
Funktio päivittää suorituksensa
lopuksi kutsupinoa siten,
että seuraavaksi kutsuttava
funktio on päälimmäisenä.
Nyt ajoympäristö voi kutsua
pinon päällä olevaa funktiota,
pitää tauon, jonka aikana
luetaan syötteet ja tarvittaessa
päivitetään piirtopinta,
ja kutsua taas hetken kuluttua
pinon päälimmäistä funktiota.
Kun funktiopinossa ei ole
enää alkioita, ohjelman suoritus
on päättynyt.

Atomiset osat pitävät vain
itsensä sisällä tarvittavat
arvot paikallisissa muuttujissa.
Muissa osissa mahdollisesti tarvittavat
arvot tallennetaan muuttujille
varattuun pinomuistiin
(huom. eri kuin kutsupino).


\subsection{Käyttöliittymä}
EppaBasicin käyttöliittymä on toteutettu
kokonaan käytten HTML5:n tarjoamia elementtejä.
Käyttöliittymän elementtien tyylit ja asettelu
on määritelty käyttämällä CSS3:a ja
toiminnallisuus on toteutettu
JavaScriptillä.

EppaBasicin käyttöliittymää ohjaava JavaScript-koodi
on jaettu moduuleihin, joista jokainen vastaa
tiettyä käyttöliitymän osaa.
Omat moduulinsa on esimerkiksi kirjautumisruudulla,
tiedostodialogeilla, käyttöoppaalla ja ilmoituksilla.
Ohjausmoduulit ovat yhteydessä HTML-näkymään
jQuery-kirjaston \cite{jquery} avulla,
mikä yksinkertaistaa ohjainten rakennetta huomattavasti.

Käyttöliittymän palvelimeen yhteydessä olevat
moduulit käyttävät Ajax-pyyntöjä
(Asyn\-chro\-nous JavaScript + XML).
Tällaisia moduuleja ovat esimerkiksi
kirjautumisen hallinta ja tiedostojen tallentaminen.

Käyttöliittymän tarjoama ohjekirja on kirjoitettu
käyttämen Markdown-merkkausta \cite{markdown}.
Käyttäjän navigoidessa ohjekirjasivulle
JavaScript-koodi hakee Ajax:in avulla palvelimelta
oikean ohjekirjasivun Markdown-muodossa,
joka muutetaan selaimen päässä HTML-muotoon käyttäen
marked-kirjastoa \cite{marked}
Näin ohjekirja voidaan tarvittaessa myöhemmin
myös tulostaa ja palvelimen ei tarvitse tehdä
yhtä paljon työtä kuin jos
muunnos tehtäisiin palvelimella.

Ohjelmakoodin muokkaamiseen EppaBasic käytettää
Ace\hyp tekstinmuokkauskomponenttia \cite{ace_about}.
Komponentti mahdollistaa omien syntaksiväritysten luomisen.
Syntaksiväritys selkeyttää koodin lukemista,
sillä avainsanojen erottaminen helpottuu.
Ace mahdollistaa myös koodin kääntämisen taustalla,
mitä EppaBasicissa käytetään kertomaan käyttäjälle
tämän tekemistä virheistä jo koodin kirjoittamisen aikana.
Teknisesti tämä on toteutettu suorittamalla
kääntäjän ensivaihe mahdollisuuksien mukaan
taustasäikeessä web workerin \cite{w3c_web_worker} avulla.
Ensivaihe kertoo käyttöliittymälle löytämistään virheistä,
jotka käyttöliittymä näyttää käyttäjälle käyttämällä
Acen huomautuksia (eng. annotations).

\subsection{Palvelin}
EppaBasic-palvelin jakaa kehitysympäristön tiedostoja ja
vastaa käyttöliittymän Ajax-pyyntöihin vastaamisesta. Ensimmäisellä
tasolla pyyntö menee Apache-palvelimelle (yleisesti käytössä oleva WWW-palvelin).
Apache-palvelin jakaa staattiset tiedostot ja ohjaa dynaamiset pyynnöt
eteenpäin sovelluspalvelimelle (eng. application server).

Sovelluspalvelin on toteutttu Django-sovelluskehystä (eng. framework)
hyödyntäen. Django perustuu moderniin MVC-malliin (Model View Controller),
joka helpottaa tietoturvallisten sivustojen kehitystä.
Sovelluspalvelin käsittelee dynaamiset Ajax-pyynnöt
varmistaen, että pyytäjällä on oikeus pyytämiinsä tietoihin,
ja palauttaa vastauksen JSON-muodossa. Pysyvänä tiedon
tallennuspaikkana, johon tallennetaan esimerkiksi rekisteröityjen
käyttäjien koodit, toimii SQL-tietokanta (Structured Query Language).

\subsection{Suoritusympäristö}
EppaBasicin suoritusympäristö on erillinen verkkosivu,
joka tarjoaa kaikille EppaBasicilla kirjoitetuille
ohjeilmille yhteiset toiminnot, kuten piirto-,
merkkijono- ja matematiikkafunktiot sekä
koodin suorittamisen ajastamiseen tarvittavat toiminnot.
Myös suoritusympäristön voi laskea kuuluvaksi käyttöliittymään,
mutta esittelen sen erillään muusta käyttöliittymästä,
sillä se tarjoaa käyttöliittymänä vain piirtopinnan
ja syötteiden lukemisen, mutta ohjelmallisena
komponenttina rajapinnan ja suoritusalusta kääntäjän
kääntämien ohjelmien suorittamiselle.

Suoritusympäristö aukeaa automaattisesti uuteen selainikkunaan
käyttäjän suorittaessa ohjelman ja koodin käännyttyä.
Suoritusympäristö pyytää muokkausikkunalta käännetyn kooodin,
minkä jälkeen suoritusympäristö alustaa itsensä ja suorittaa koodin.

Suoritusympärstö koostuu kahdesta osasta:
piirtäjästä ja työskentelijästä.
Piirtäjä on selaimessa pyörivä
JavaScript-ohjelma, joka huolehtii
näytön piirtämisestä ja käyttäjän syötteiden lukemisesta.
Työskentelijä suoritetaan omassa säikeessään.
Näin saadaan ohjelman suoritus eristettyä
irralleen käyttöliittymätoiminnoista.
Jos näin ei tehtäisi, ohjelman suorituksen
kestäessä selain ei vastaisi käyttäjän
syötteisiin ennen suorituksen taukoamista.

Piirtäjä ja työskentelijä keskustelevat
keskenään käyttäen viestejä.
Viestit voivat olla yksinkertaisia
JavaScript-merkkijonoja,
-olioita, -taulukoita ja -lukuja
sekä näiden yhdistelmiä.
Viestit kopioituvat lähetettäessä,
joten arvojen muuttaminen piirtäjässä
ei vaikuta työskentelijän vastaanottamaan
arvoon eikä myöskään toisinpäin.
Viestejä voi lähettää mistä tahansa kohtaa
suoritettavaa JavaScript-koodia,
mutta niiden lukemiseksi suoritus on katkaistava.
Tämä ominaisuus on suurin syy kääntäjän
tavallista monimutkaisempaan rakenteeseen.

\subsubsection{Piirtäjä}
Piirtäjä on suoritusympäristön komponentti,
joka nimensä mukaisesti huolehtii
grafiikan piirtämisestä,
mutta myös käyttäjän syötteiden kirjaamisesta.
Grafiikan piirtämiseen käytetään
HTML5 Canvas -elementtiä,
joka on nykyään tuettu kaikissa moderneissa
selaimissa \cite{caniuse_canvas}.
Canvas mahdollistaa yksinkertaisten
geometristen muotojen piirtämisen tehokkaasti,
mutta myös kuvien käyttäminen on mahdollista.
Ohjeet grafiikan piirtämiseen piirtäjä
saa viesteinä työskentelijältä.
Yksi viesti sisältää useita piirtokomentoja,
jotka kaikki suoritetaan taustapuskuriin ennen
canvas-elementin päivittämistä.
Tätä kutsutaan kaksoispuskuroinniksi
(eng. double-buffering).

Syötteen (näppäimistö ja hiiri) lukemiseksi
piirtäjä kuuntelee kaikkia suoritusympäristön
näp\-päin- ja hiiritapahtumia (eng. events).
Tapahtuman sattuessa piirtäjä
lähettää tiedon muutoksesta työskentelijälle,
joka päivittää omaa tilaansa.

\subsubsection{Työskenteljiä}
Työskentelijä on suoritusympäristön osa,
joka suorittaa käännetyn koodin.
Työskentelijä ajetaan taustasäikeessä
web workerin avulla, jolloin ohjelman
suoritus ei häiritse selaimen muuta toimintaa.
Työskentelijä tarjoaa EppaBasic-ohjelmille
rajapinnan, jonka avulla ne voivat käyttää
standardikirjaston funktioita.
Suurin osa näistä funktioista on toteutettu
muuttamalla JavaScriptin tarjoamat valmiit
funktiot EppaBasicin ymmärtämään muotoon.
Ne funktiot, joita JavaScript ei suoraan tarjoa,
on toteutettu joko käyttämällä kolmannen osapuolen
tarjoamaa toteutusta tai yksinkertaisissa
tapauksissa toteuttamalla itse.

Työskentelijä pitää sisäistä tilaa,
joka sisältää esimerkiksi nykyisen piirtovärin
sekä näppäimistön pohjassa olevat painikkeet.
Sisäinen tila voi muuttua kahdella tavalla:
joko EppaBasic-koodissa kutsutaan komentoa,
joka muuttaa sisäistä tilaa
(esimerkiksi piirtovärin asettaminen),
tai piirtäjä lähettää viestin,
joka muuttaa sisäistä tilaa
(esimerkiksi hiiren sijainti).
Sisäinen tila vaikuttaa joidenkin
EppaBasicin komentojen palauttamiin arvoihin
(esimerkiksi \eb{KeyDown} ja \eb{MouseX})
sekä piirtokomentojen yhteydessä määrittämään
piirtokomennon tarkkaa toimintaa
(esimerkiksi viivan paksuus tai väri).
