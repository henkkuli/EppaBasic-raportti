% Täällä kerrotaan sivuston sekä kääntäjän teknisestä toteutuksesta.

\section{Tekninen toteutus}
\fxnote{Miten toteutettu?}

Jotta EppaBasic toimisi mahdollisimman monella tietokoneella käyttöjärjestelmästä riippumatta,
päätimme toteuttaa kaiken ohjelmointi- ja suoritusympäristöä myöten selainpohjaisesti.
Näin ohjelmointikielemme käyttäminen onnistuu niin Windows-, Linux- ja OS X -koneilla
kuin myös monilla mobiililaitteilla.
Lisäksi modernit selaimet tarjoavat monia hyödyllisiä ominaisuuksia
kuten yksinkertaisen tavan toteuttaa käyttöliittymiä (HTML \cite{w3c_html})
sekä helpon tavan piirtää grafiikkaa näytölle (HTML Canvas \cite{mdn_canvas}).
Koko ohjelmakoodi EppaBasicin takana on kirjoitettu JavaScriptillä,
joka on ainoa yleisesti selaimissa tuettu ohjelmointikieli \cite{mdn_about_js}.

\subsection{Käyttöliittymä}
\begin{anfxnote}
Tässä kerrotaan käyttöliittymän rakenteesta.
Pääasiassa kerrotaan, että on toteutettu html:llä sekä javascriptillä.
Lisäksi kerrotaan kirjautumismahdollisuudesta sekä
sen tuomista eduista (tallentaminen, jakaminen).
\end{anfxnote}

EppaBasicin käyttöliittymä on toteutettu kokonaan käyttäen HTMLää sekä JavaScriptiä.
Näin on saatu aikaa kaikissa moderneissa selaimissa toimiva yhtenäinen käyuttöliittymä.

EppaBasicin käyttöliittymä sisältää kaikki ohjelmoinnin kannalta tarvittavat toiminnallisuudet:
Yläpalkista löytyy ohjelman suorittamiseen tarvittava "Käynnistä"-näppäin,
tiedostojen hallintaan tarvittavat ''Uusi''-, ''Lataa''- ja ''Tallenna''-näppäimet,
koodin jakamiseen käytettävä ''Jaa koodi''-näppäin,
kielen valinta
sekä kirjautumiseen tarvittavat kentät.
Lisäksi käyttöliittymästä löytyy vasemmalta puolelta suuri värittävä koodimuokkain
sekä oikealta puolelta löytyy sivuston uutiset sekä käyttöohje.
Tällaisia ohjelmia kutsutaan nimellä Integrated Development Environment (IDE).

Kirjautumiskenttää käyttäen rekisteröityneet käyttäjät voivat kirjautua sivustolle.
Kirjautumalla saa käyttöönsä koodin tallentamismahdollisuuden.
Tallennetut koodit tallennetaa palvelimelle
ja ne on käytettävissä kaikilla koneilla.
Kirjautumattomat käyttäjät eivät voi tallentaa koodejaan sivustolle,
mutta he voivat ladata esimerkkikoodeja käyttäen lataa-toimintoa.

Koodimuokkaimena käytämme Ace-tekstinmuokkauskomponenttia \cite{ace_about}.
Ace mahdollistaa koodin monipuolisen muokkaamisen
sekä omien syntaksiväritysten luomisen.
Syntaksiväritys värittää avainsanat omilla väreillään,
mikä selkeyttää koodin lukemista.



\subsection{Kääntäjä}
\begin{anfxnote}
Kääntäjään todellisesta toteutuksesta olisi tarkoitus kertoa tässä.
\end{anfxnote}

\subsection{Ajoympäristö}
\begin{anfxnote}
Tässä on tarkoitus kertoa uudesta ajoympäristöstä.
Toimii webworkereilla taustasäikeessä.
Suoritusta katkotaan kaksisuuntaisen liikenteen sallimiseksi.
Toiseen suuntaa kulkee grafiikkakomennot ja toiseen suuntaan
käyttäjän syötteet sekä joskus myöhemmin mahdollisesti
myös debug-komennot.
\end{anfxnote}

\subsection{Kieli}
\begin{anfxnote}
Tässä kerrotaan kielen ominaisuuksista
sekä sen vahvoista puolista.
\end{anfxnote}
