% Täällä kerrotaan sivuston sekä kääntäjän teknisestä toteutuksesta.

\section{Tekninen toteutus}
\fxnote{Miten toteutettu?}

Jotta EppaBasic toimisi mahdollisimman monella tietokoneella käyttöjärjestelmästä riippumatta,
päätimme toteuttaa kaiken ohjelmointi- ja suoritusympäristöä myöten selainpohjaisesti.
Näin ohjelmointikielemme käyttäminen onnistuu niin Windows-, Linux- ja OS X -koneilla
kuin myös monilla mobiililaitteilla.
Lisäksi modernit selaimet tarjoavat monia hyödyllisiä ominaisuuksia
kuten yksinkertaisen tavan toteuttaa käyttöliittymiä (HTML \cite{w3c_html})
sekä helpon tavan piirtää grafiikkaa näytölle (\eng{HTML Canvas} \cite{mdn_canvas}).
Koko ohjelmakoodi EppaBasicin takana on kirjoitettu JavaScriptillä,
joka on ainoa yleisesti selaimissa tuettu ohjelmointikieli \cite{mdn_about_js}.

\subsection{Käyttöliittymä}
\begin{anfxnote}{}
Tässä kerrotaan käyttöliittymän rakenteesta.
Pääasiassa kerrotaan, että on toteutettu html:llä sekä javascriptillä.
Lisäksi kerrotaan kirjautumismahdollisuudesta sekä
sen tuomista eduista (tallentaminen, jakaminen).
\end{anfxnote}

EppaBasicin käyttöliittymä on toteutettu kokonaan käyttäen HTMLää sekä JavaScriptiä.
Näin on saatu aikaa kaikissa moderneissa selaimissa toimiva yhtenäinen käyttöliittymä.

EppaBasicin käyttöliittymä sisältää kaikki ohjelmoinnin kannalta tarvittavat toiminnallisuudet:
Yläpalkista löytyy ohjelman suorittamiseen tarvittava "Käynnistä"-näppäin,
tiedostojen hallintaan tarvittavat ''Uusi''-, ''Lataa''- ja ''Tallenna''-näppäimet,
koodin jakamiseen käytettävä ''Jaa koodi''-näppäin,
käyttöliittymän kielen valinta
sekä kirjautumiseen tarvittavat kentät.
Lisäksi käyttöliittymästä löytyy vasemmalta puolelta suuri värittävä koodimuokkain
sekä oikealta puolelta löytyy sivuston uutiset sekä käyttöohje.
Tällaisia ohjelmia kutsutaan nimellä Integrated Development Environment (IDE).

Kirjautumiskenttää käyttäen rekisteröityneet käyttäjät voivat kirjautua sivustolle.
Kirjautumalla saa käyttöönsä koodin tallentamismahdollisuuden.
Tallennetut koodit tallennetaa palvelimelle
ja ne on käytettävissä kaikilla koneilla.
Kirjautumattomat käyttäjät eivät voi tallentaa koodejaan sivustolle,
mutta he voivat ladata esimerkkikoodeja käyttäen lataa-toimintoa.

Koodimuokkaimena käytämme Ace-tekstinmuokkauskomponenttia \cite{ace_about}.
Ace mahdollistaa koodin monipuolisen muokkaamisen
sekä omien syntaksiväritysten luomisen.
Syntaksiväritys värittää avainsanat omilla väreillään,
mikä selkeyttää koodin lukemista.

Lisäksi Ace mahdollistaa yhdessä kääntäjän kanssa käyttäjän tekemien virheiden näyttämisen jo kirjoittamisen aikana.
Näin käyttäjä saa välitöntä palautetta valitsemallaan omalla äidinkielellään koodissa olevista virheistä ilman,
että tarvitsee ensin suorittaa koodia.
Tämä parantaa ohjelmoijan työskentelyn sujuvuutta.

Teknisesti edellä mainittu ominaisuus on toteutettu siten,
että aina muokkaimen tekstin muuttuessa suoritetaan kääntäjän front-end,
joka tunnistaa kaikki syntaksi- ja tyyppivirheet.
Testikääntäminen suoritetaan mahdollisuuksien mukaan taustasäikeessä (\eng{Web Worker} \cite{w3c_web_worker}).
Näin virheiden tarkistaminen ei hidasta koodin kirjoittamista.

\subsection{Kääntäjä}
\begin{anfxnote}{}
Kääntäjään todellisesta toteutuksesta olisi tarkoitus kertoa tässä.
\end{anfxnote}

\subsection{Ajoympäristö}
\begin{anfxnote}{}
Tässä on tarkoitus kertoa uudesta ajoympäristöstä.
Toimii webworkereilla taustasäikeessä.
Suoritusta katkotaan kaksisuuntaisen liikenteen sallimiseksi.
Toiseen suuntaa kulkee grafiikkakomennot ja toiseen suuntaan
käyttäjän syötteet sekä joskus myöhemmin mahdollisesti
myös debug-komennot.
\end{anfxnote}

Kun käyttäjä painaa käyttöliittymän ''käynnistä''-nappia,
käyttöliittymä avaa käännöksen jälkeen selaimeen uusi ikkuna,
ajoympäristön.
Ajoympäristö pyytää käynnistyessään käyttöliittymältä kääntäjän kääntämää koodia,
minkä jälkeen ajoympäristö suorittaa koodin.
Ajoympäristö tarjoaa kaikille ohjelmille yhteiset toiminnot,
kuten piirto-, merkkijono- ja matematiikkafunktiot
sekä koodin suorittamisen ajastamisen tarvittavat toiminnot.

Ajoympäristö -- kuten koko käyttöliittymä muutenkin --
on totteutettu käyttäen JavaScriptiä.
Ajoympäristö koostuu kahdesta osasta:
piirtäjästä ja työskentelijästä.
Piirtäjä on tavallinen \eng{JavaScript}-ohjelma,
joka huolehtii näytön piirtämisestä sekä käyttäjän syötteiden lukemisesta.
Ohjelman suorituksen käynnistyessä piirtäjä alustaa myös työskentelijän,
joka pyörii omassa säikeessään käyttäen \eng{Web Worker}ia.
Näin saadaan ohjelman suoritus eristettyä irralleen selaimen toiminnasta,
jolloin ohjelman suorituksen kestäessä kauan selain pysyy toimintakunnossa.

Piirtäjä ja työskentelijä keskustelevat keskenään käyttäen viestejä.
Viestit voivat olla yksinkertaisia \eng{JavaScript}-merkkijonoja,
-objekteja, -taulukoita ja -lukuja sekä näiden yhdistelmiä.
Viestit kopioituvat lähettäessä, joten arvojen muuttaminen
piirtäjässä ei vaikuta työskentelijän vastaanottamaana arvoon.
Viestejä voi lähettää missä tahansa kohtaa suoritettavaa
\eng{JavaScript}-koodia, mutta niiden lukemiseksi
suoritus on katkaistava.
Tämä ominaisuus on syynä kääntäjän tavallista monimutkaisempaan rakenteeseen.

\subsubsection{Piirtäjä}
Piirtäjä käyttää grafiikan piirtämiseen \eng{HTML5 Canvas} -elementtiä,
joka on nykyään tuettu kaikissa moderneissa selaimissa
\cite{caniuse_canvas}.
Canvas mahdollistaa yksinkertaisten geometristen muotojen piirtämisen tehokkaasti,
mutta myös kuvien käyttäminen on mahdollista.
Käyttäjän syötteen (näppäimistö ja hiiri) lukemiseksi
ohjelma kuuntelee ajoympäristön kaikkia näppäin- ja hiiritapahtumia (\eng{Events}).
Tapahtuman sattuessa piirtäjä lähettää muutoksen työskentelijälle,
joka päivittää omaa tilaansa.
Piirtäjä piirtää näytön työskentelijältä saamiensa ohjeiden mukaan.

\subsubsection{Työskentelijä}
Työskentelijällä on sisäinen tila,
joka sisältää esimerkiksi nykyisen piirtovärin
sekä näppäimistön pohjassa olevat painikkeet.
Sisäinen tila voi muuttua kahdella tavalla:
joko EppaBasic-koodissa kutsutaan komentoa,
joka muuttaa sisäistä tilaa (esimerkiksi piirtovärin asettaminen),
tai piirtäjä lähettää komennon,
joka muuttaa sisäistä tilaa (esimerkiksi hiiren sijainti).
Sisäinen tila vaikuttaa joidenkin EppaBasicin komentojen palauttamiin arvoihin
(esimerkiksi \eb{KeyDown} ja \eb{MouseX})
sekä piirtokomentojen yhteydessä määrittämään
piirtokomennon tarkka toiminta (esimerkiksi viivan paksuus tai väri).

Suurin osa matematiikka-, merkkijono sekä ajanhallintafunktiosta on toteutettu käytännössä
käärimällä \eng{JavaScriptin} tarjoamat funktiot EppaBasicin tarvitsemaan muotoon.
Ne funktiot, joita \eng{JavaScript} ei suoraan tarjoa, on toteutettu
joko käyttämällä kolmannen osapuolen tarjoaamaa kirjastoa
tai yksinkertaisissa tapauksissa kirjoittamalla itse.

Grafiikkafunktiot on toteutettu siten, että ne kerätään sisäiseen piirtojonoon.
Kutsuttaessa \eb{DrawScreen}-komentoa piirtojono lähetetään piirtäjälle.
Näin saadaan varmistettua, että piirtäjä piirtää kaiken kerrallaan
eikä piirroksen osia ilmesty ruudulle sattumanvaraisesti.

\subsection{Kieli}
\begin{anfxnote}{}
Tässä kerrotaan kielen ominaisuuksista
sekä sen vahvoista puolista.
\end{anfxnote}
