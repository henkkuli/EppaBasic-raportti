% Täällä kerrotaan sivuston sekä kääntäjän teknisestä toteutuksesta.

\section{Tekninen toteutus}
\fxnote{Miten toteutettu?}

Jotta EppaBasic toimisi mahdollisimman monella tietokoneella käyttöjärjestelmästä riippumatta,
päätimme toteuttaa kaiken ohjelmointi- ja suoritusympäristöä myöten selainpohjaisesti.
Näin ohjelmointikielemme käyttäminen onnistuu niin Windows-, Linux- ja OS X -koneilla
kuin myös monilla mobiililaitteilla.
Lisäksi modernit selaimet tarjoavat monia hyödyllisiä ominaisuuksia
kuten yksinkertaisen tavan toteuttaa käyttöliittymiä (HTML \cite{w3c_html})
sekä helpon tavan piirtää grafiikkaa näytölle (HTML Canvas \cite{mdn_canvas}).
Koko ohjelmakoodi EppaBasicin takana on kirjoitettu JavaScriptillä,
joka on ainoa yleisesti selaimissa tuettu ohjelmointikieli \cite{mdn_about_js}.

\subsection{Käyttöliittymä}
\begin{anfxnote}
Tässä kerrotaan käyttöliittymän rakenteesta.
Pääasiassa kerrotaan, että on toteutettu html:llä sekä javascriptillä.
Lisäksi kerrotaan kirjautumismahdollisuudesta sekä
sen tuomista eduista (tallentaminen, jakaminen).
\end{anfxnote}

	
\subsection{Kääntäjä}
\begin{anfxnote}
Kääntäjään todellisesta toteutuksesta olisi tarkoitus kertoa tässä.
\end{anfxnote}

\subsection{Ajoympäristö}
\begin{anfxnote}
Tässä on tarkoitus kertoa uudesta ajoympäristöstä.
Toimii webworkereilla taustasäikeessä.
Suoritusta katkotaan kaksisuuntaisen liikenteen sallimiseksi.
Toiseen suuntaa kulkee grafiikkakomennot ja toiseen suuntaan
käyttäjän syötteet sekä joskus myöhemmin mahdollisesti
myös debug-komennot.
\end{anfxnote}
