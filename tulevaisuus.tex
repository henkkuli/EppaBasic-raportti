\section{Tulevaisuus}
EppaBasicin kehitystä on tarkoitus
jatkaa myös tulevaisuudessa.
Tämän raportin kirjoitushetkellä työn alla
on uusi kääntäjä,
jonka tarkoitus on parantaa käännöksen laatua.
Myöhemmin on lisäksi tarkoitus lisätä
mahdollisuus piirtää bittikarttakuvia
ja toistaa ääntä.
Lähitulevaisuudessa EppaBasicista on tarkoitus
saada varteenotettava vaihtoehto niin ohjelmoinnin
kouluopettamiseen kuin itseopiskeluunkin.
Kehitystietä on valitettavasti hidastanut kehittäjien
lukio-opiskelu sekä lähestyviin ylioppilaskokeisiin
valmistautuminen.

EppaBasicia on tarkoitus käyttää keväällä
ja kesällä 2015 ainakin kahdella kurssilla:
Keväällä Helsingin luonnontiedelukiossa
järjestetään ohjelmoinnin peruskurssi,
joka opetetaan kokonaisuudessaan
EppaBasicilla.
Kesällä kieltä käytetään jälleeen
Helsingin yliopiston LUMA-keskuksen
"Bittejä ja algoritmeja" -leirillä.