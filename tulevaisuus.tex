\section{Tulevaisuus}
EppaBasicin kehitystä on tarkoitus
jatkaa myös tulevaisuudessa.
Tämän raportin kirjoitushetkellä työn alla
on uusi kääntäjä,
jonka tarkoitus on parantaa käännöksen laatua.
Myöhemmin on suunnitelmissa lisätä
mahdollisuus ladata ja piirtää kuvia tiedostoista
sekä toistaa ääntä.
Lähitulevaisuudessa EppaBasicista on tarkoitus
saada varteenotettava vaihtoehto niin ohjelmoinnin
kouluopettamiseen kuin itseopiskeluunkin.
Kehitystyötä on valitettavasti hidastanut kehittäjien
lukio-opiskelu sekä lähestyviin ylioppilaskokeisiin
valmistautuminen.

EppaBasicia on tarkoitus käyttää keväällä
ja kesällä 2015 ainakin kahdella kurssilla:
Keväällä Helsingin luonnontiedelukiossa
järjestetään ohjelmoinnin peruskurssi,
joka opetetaan kokonaisuudessaan
EppaBasicilla.
Kesällä kieltä käytetään jälleen
Helsingin yliopiston LUMA-keskuksen
''Bittejä ja algoritmeja'' -leirillä.