% Yleistä höpinää kääntäjien toiminnasta sekä käsitteiden selityksiä

\section{Teoriaa ja määritelmiä}
Kääntäjän toiminta voidaan jakaa pääasiallisesti kolmeen päävaiheeseen:
\eng{front-end}, välivaihe ja \eng{back-end}.
Seuraavaksi selitetään, mitä näillä vaiheilla tarkoitetaan
sekä määritellään muita raportissa käytetäviä käsitteitä.


\subsection{Front-end}
Front-end on kääntämisen ensimmäinen päävaihe.
Se lukee käyttäjän kirjoittaman lähdekoodin
ja muuttaa sen kääntäjän sisäiseen muotoon.
Front-end on ainoa osa, joka käsittelee ohjelmaa kokonaisuutena
ja tietää ohjelmointikielen syntaksin.
Muut vaiheet eivät välttämättä ole riippuvaisia käännettävästä
kielestä vaan voivat toimia front-endistä riippumatta.
Front-endin tehtäviin kuuluu myös käännöksen aikaisista virheistä tiedottaminen.

\subsection{Välivaihe}
Välivaihe on kääntämisen toinen päävaihe.
Sen tarkoitus on muokata ohjelmaa jollakin tavalla paremmaksi.
Yksinkertaisimmillaan välivaihe vain muuttaa ohjelman sisäisestä
muodosta toiseen.
Useissa kääntäjissä on monia välivaiheita, joita saatetaan jopa
iteroida useita kertoja paremman lopputuloksen saamiseksi.
Paremmalle ei tässä yhteydessä ole yksikäsitteistä määritelmää,
vaan se saattaa eri yhteyksissä tarkoittaa esimerkiksi
nopeampaa suoritusta, pienempää käännetyn ohjelman kokoa
tai pienempää muistinkäyttöä. Useimmiten suorituksen nopeuttaminen
on pääintressi tietokoneohjelmissa, kun taas muistin käytön
vähentäminen on tärkeää sulautetuissa järjestelmissä.
Täälaista ohjelmakoodin parantamista kutstuaan optimoimiseksi,
ja vastaavasti sitä tekeviä välivaiheta optimoijiksi.

\subsection{Back-end}
\eng{Back-end} on kääntämisen viimeinen vaihe.
Sen tehtävä on muutta ohjelmakoodi lopulliseen muotoonsta,
jota ajamiseen käytetty ympäristö ymmärtää.
Ajoympäristö on usein joko suoraan rautatason laite
tai virtuaalikone, joka ajon aikana muuttaa koodin raudan ymmärtämään muotoon
tai vaihtoehtoisesti tulkitsee koodia.
\eng{Back-end} on periaatteessa kääntämisen ainoa vaihe,
joka välittää käännöksen kohteen arkkitehtuurista,
tosin joskus myös viimeiset välivaiheen vaiheet riippuvat kohdearkkitehtuurista.

\subsection{Syntaksi}
Syntaksi tarkoittaa kielen kielioppia.
Ohjelmoidessa sen on tarkoitus olla täysin yksiselitteinen,
jottei kääntäjän tarvitse arvuutella,
mitä ohjelmoija haluaa ohjelman tekevän.
Syntaksi esimerkiksi määrittelee,
että $If$-avainsanaa on seurattava $Boolean$-tyypiksi evaluoituva lauseke
tai mitkä nimet ovat sallittuja muuttujannimiä.

\subsection{Lähdekoodi}
Lähdekoodi on käyttäjän kirjoittama kuvaus ohjelman toiminnasta.
Kääntäjä muuttaa lähdekoodin kohdekoodiksi,
jota tietokone, virtuaalikone tai tulkki suorittaa.
Lähdekoodi on usein ihmisen helposti ymmärrettävissä
sekä sisältää monia ymmärtämistä helpottavia abstraktioita,
kuten esimerkiksi muuttujat ja komentorakenteet.
Lähdekoodi noudattaa syntaksia.

\subsection{Ohjelmointikieli}
Ohjelmointikieli on syntaksin määrittämä kokonnaisuus.
Ohjelmointikielet voidaan luonnollisten kielien
(suomi, ruotsi, yms.)
tavoin jakaa kielisukuihin,
usein suvun ensimmäisen kielen mukaan.
Ohjelmointikielisukuja ovat esimerkiksi
C-pohjaiset ja
Basic-kielet.
