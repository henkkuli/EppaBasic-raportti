% Yleistä höpinää kääntäjien toiminnasta sekä käsitteiden selityksiä

\section{Teoriaa ja määritelmiä}
Kääntäjän toiminta voidaan jakaa pääasiallisesti kolmeen päävaiheeseen:
front-end, välivaihe ja back-end.
Seuraavaksi selitetään, mitä näillä vaiheilla tarkoitetaan
sekä määritellään muita raportissa käytetäviä käsitteitä.


\subsection{Front-end}
Front-end on kääntämisen ensimmäinen päävaihe.
Se lukee käyttäjän kirjoittaman lähdekoodin
ja muuttaa sen kääntäjän sisäiseen muotoon.
Front-end on ainoa osa, joka käsittelee ohjelmaa kokonaisuutena
ja tietää ohjelmointikielen syntaksin.
Muut vaiheet eivät välttämättä ole riippuvaisia käännettävästä
kielestä vaan voivat toimia front-endistä riippumatta.
Front-endin tehtäviin kuuluu myös käännöksen aikaisista virheistä tiedottaminen.

\subsection{Välivaihe}
Välivaihe on kääntämisen toinen päävaihe.
Sen tarkoitus on muokata ohjelmaa jollakin tavalla paremmaksi.
Yksinkertaisimmillaan välivaihe vain muuttaa ohjelman sisäisestä
muodosta toiseen.
Useissa kääntäjissä on monia välivaiheita, joita saatetaan jopa
iteroida useita kertoja paremman lopputuloksen saamiseksi.
Paremmalle ei tässä yhteydessä ole yksikäsitteistä määritelmää,
vaan se saattaa eri yhteyksissä tarkoittaa esimerkiksi
nopeampaa suoritusta, pienempää käännetyn ohjelman kokoa
tai pienempää muistinkäyttöä. Useimmiten suorituksen nopeuttaminen
on pääintressi tietokoneohjelmissa, kun taas muistin käytön
vähentäminen on tärkeää sulautetuissa järjestelmissä.

Optimoinnit...

\subsection{Back-end}
\subsection{Syntaksi}
\subsection{Lähdekoodi}
\subsection{Ohjelmointikieli}