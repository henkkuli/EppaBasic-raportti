% Yleistä höpinää kääntäjien toiminnasta sekä käsitteiden selityksiä

\section{Teoriaa ja määritelmiä}
Tietokoneet ymmärtävät konekieltä.
Konekieli muodostuu yksinkertaisisista komennoista,
joiden nopeaan suorittamiseen tietokoneet on suunniteltu.
Ihmiset ovat kuitenkin huonoja kirjoittamaan konekieltä
sen alkukantaisten komentojen vuoksi.
Ihmisiä varten onkin kehitelty erilaisia ohjelmointikieliä,
jotka sisältävät erilaisia abstraktioita ohjelmoinnin helpottamiseksi.

Muunnos ohjelmointikielestä konekieleen ei kuitenkaan aina ole yksinkertainen.
Monet ohjelmointikielten abstraktiot voidaan toteuttaa hyvinkin eri tavoin konekielessä,
sillä eri tietokonearkkitehtuurit sisältävät hyvinkin erilaisia konekielisiä komentoja.
Lisäksi lyhyistäkin ohjelmakoodinpätkistä voi syntyä pitkiä konekielisiä ohjelmia,
joten pientenkin ohjelmien muuttaminen konekieleksi voi viedä ohjelmoijalta tunteja.

Tämän prosessin automatisoimiseksi on kehitetty ohjelmia, joita kutsutaan kääntäjiksi.
Kääntäjä on yksinkertaisesti ohjelma, joka muuttaa koodia muodosta toiseen.
Usein tämä tarkoittaa ohjelmointikielen muuttamista konekieleksi,
mutta on myös olemassa kääntäjiä,
jotka kääntävät ohjelmointikieliä toisikseen.
Tällaisia kääntäjiä kutsutaan source-to-source-kääntäjiksi.

Kääntäjän toiminta voidaan jakaa pääasiallisesti kolmeen päävaiheeseen:
front-end, välivaihe ja back-end.
Seuraavaksi selitetään, mitä näillä vaiheilla tarkoitetaan
sekä määritellään muita raportissa käytetäviä käsitteitä.


\subsection{Front-end}
Front-end on kääntämisen ensimmäinen päävaihe.
Se lukee käyttäjän kirjoittaman lähdekoodin
ja muuttaa sen kääntäjän sisäiseen muotoon.
Front-end on ainoa osa, joka käsittelee ohjelmaa kokonaisuutena
ja tietää ohjelmointikielen syntaksin.
Muut vaiheet eivät välttämättä ole riippuvaisia käännettävästä
kielestä vaan voivat toimia front-endistä riippumatta.
Front-endin tehtäviin kuuluu myös käännöksen aikaisista virheistä tiedottaminen.

\subsection{Välivaihe}
Välivaihe on kääntämisen toinen päävaihe.
Sen tarkoitus on muokata ohjelmaa jollakin tavalla paremmaksi.
Yksinkertaisimmillaan välivaihe vain muuttaa ohjelman sisäisestä
muodosta toiseen.
Useissa kääntäjissä on monia välivaiheita, joita saatetaan jopa
iteroida useita kertoja paremman lopputuloksen saamiseksi.
Paremmalle ei tässä yhteydessä ole yksikäsitteistä määritelmää,
vaan se saattaa eri yhteyksissä tarkoittaa esimerkiksi
nopeampaa suoritusta, pienempää käännetyn ohjelman kokoa
tai pienempää muistinkäyttöä. Useimmiten suorituksen nopeuttaminen
on pääintressi tietokoneohjelmissa, kun taas muistin käytön
vähentäminen on tärkeää sulautetuissa järjestelmissä.

Optimoinnit...

\subsection{Back-end}
\subsection{Syntaksi}
\subsection{Lähdekoodi}
\subsection{Ohjelmointikieli}