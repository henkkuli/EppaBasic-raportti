% Täällä kerrotaan kielen suunnitettelulähtökohdista
\section{Tavoitteet}
Omalle projektillemme asetimme monia tavoitteita.
Näistä tärkeimpänä voidaan pitää yksinkertaisuutta
sekä matalaa oppimiskäyrää.
Kielellä täytyy myös voida helposti luoda grafiikkaa,
sillä monien aloittelijoiden ohjelmointi-into loppuu siihen,
että graafisesti näyttävien pelien tekeminen ei onnistukaan terminaalissa.

Toinen tärkeä tavoite on,
että kielen käyttöönotto on mahdollisimman helppoa.
Usein uutta ohjelmointikieltä käyttöönottaessa
kuluu tunteja aikaa ohjelmointiympäristön asentamiseen.
Aloittelijoille tämä on usein lannistavaa,
mikä pysäyttää ohjelmointihalut alkumetreilleen.
Halusimme myös, että ohjelmointikieli toimii mahdollisimman monella käyttöjärjestelmällä.

Ennen projektin aloittamista mietimme,
täyttäisikö jokin olemassa oleva kieli jo nämä tavoitteet.
Kuitenkin miettiessämme vaihtoehtoja
olivat useat kielet joko vanhentuneita,
niiden asentaminen tai oppiminen on hankalaa,
tai niillä grafiikan tekeminen on työlästä.
Tämän takia päädyimme uuden kielen tekemiseen.

\begin{comment}
Usein uutta ohjelmointikieltä käyttöönottaessani
minulla kuluu enemmän aikaa ohjelmointiympäristön asentamiseen
kuin uuden kielen oppimiseen.
Vaikka tämä ongelma ei suoranaisesti
\end{comment}

\begin{anfxnote}
Tänne on tarkoitus listata projektin tavoitteita.
\\
Miksi uusi ohjelmointikieli?
\\
Mitä uutta?
\\
Miten erottuu?
\end{anfxnote}