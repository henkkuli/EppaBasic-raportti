% Täällä kerrotaan kielen suunnitettelulähtökohdista
\section{Tavoitteet}
Omalle projektillemme asetimme monia tavoitteita.
Näistä tärkeimpänä voidaan pitää yksinkertaisuutta
sekä matalaa oppimiskäyrää.
Kielellä täytyy myös voida helposti luoda grafiikkaa,
sillä monien aloittelijoiden ohjelmointi-into loppuu siihen,
että graafisesti näyttävien pelien tekeminen ei onnistukaan terminaalissa.

Toinen tärkeä tavoite on,
että kielen käyttöönotto on mahdollisimman helppoa.
Usein uutta ohjelmointikieltä käyttöönottaessa
kuluu tunteja aikaa ohjelmointiympäristön asentamiseen.
Aloittelijoille tämä on usein lannistavaa,
mikä pysäyttää ohjelmointihalut alkumetreilleen.
Halusimme myös, että ohjelmointikieli toimii mahdollisimman monella käyttöjärjestelmällä.

Ennen projektin aloittamista mietimme,
täyttäisikö jokin olemassa oleva kieli jo nämä tavoitteet.
Kuitenkin miettiessämme vaihtoehtoja
olivat useat kielet joko vanhentuneita,
niiden asentaminen tai oppiminen on hankalaa,
tai niillä grafiikan tekeminen on työlästä.
Tämän takia päädyimme uuden kielen tekemiseen.

\subsection{Teknisiä vaatimuksia}
Kieltä suunnitellessamme päädyimme nopeasti Basic-tyyppisiin kieliin,
sillä niissä koodilohkojen alkujen ja loppujen merkitsemiseen käytetään englanninkielisiä sanoja
toisin kuin esimerkiksi C-pohjaisissa kielissä (C++, \eng{Java}, \eng{JavaScript}),
joissa vastaaviin tarkoituksiin käytetään aaltosulkuja.
Lisäksi monet muutkin asiat,
kuten esimerkiksi muuttujien tyyppien valitseminen,
tapahtuu käyttäen englanninkielisiä sanoja erilaisten lyhennysmerkintöjen sijaan.
Vaikka pidempien ilmaisujen kirjoittaminen onkin lyhyitä hitaampaa,
päätimme, että aloittelijoille selkeys on tärkeintä.

Välttääksemme monien kielien heikosta tyypityksestä johtuvat ongelmat
-- kuten luvun sisältämän merkkijonon ja luvun yhteenlasku tuottaa PHP-kielessä luvun \fxnote{Selvempi esimerkki} --,
päätimme, että kielessä on oltava vahva tyypitys.
Näin kääntäjä voi jo käännöksen aikana varoittaa edellä mainitun kaltaisista tapauksista,
jotka saattavat aiheuttaa vakaviakin bugeja ja
joiden löytäminen koodista saattaa olla todella hankalaa kokeneellekin ohjelmoijalle.

\begin{comment}
Usein uutta ohjelmointikieltä käyttöönottaessani
minulla kuluu enemmän aikaa ohjelmointiympäristön asentamiseen
kuin uuden kielen oppimiseen.
Vaikka tämä ongelma ei suoranaisesti
\end{comment}

\begin{anfxnote}{}
Tänne on tarkoitus listata projektin tavoitteita.
\\
Miksi uusi ohjelmointikieli?
\\
Mitä uutta?
\\
Miten erottuu?
\end{anfxnote}