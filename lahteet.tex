% Täällä listataan raportin teossa käytetyt lähteet

\begin{thebibliography}{99}

\bibitem{repl.it}
	Repl.it-sivusto \url{http://repl.it/languages} (luettu 22.10.2014)

\bibitem{eac2e}
	Keith D. Cooper \& Linda Torczon,
	\emph{Engineering a Compiler}.
	Morgan Kaufmann Publishers,
	Second Edition,
	2012.

\bibitem{basic}
	Thomas E. Kurtz,
	\emph{History of programming languages I}, Osa XI.
	ACM,
	1981.

\bibitem{pascal}
	Niklaus Wirth,
	\emph{The programming language Pascal}.
	Acta informatica,
	1971.
	
\bibitem{language_history}
	A History of Computer Programming Languages \url{http://cs.brown.edu/~adf/programming_languages.html} (luettu 24.1.2015)
	
\bibitem{OPS_2016}
	OPS 2016 - Esi- ja perusopetuksen opetussuunnitelman perusteiden uudistaminen \url{http://www.oph.fi/ops2016} (luettu 27.1.2015)

\bibitem{koodi2016_ops}
	Koodi2016 | Miten OPS muuttuu ja miksi? \url{http://koodi2016.fi/ops.html} (luettu 27.1.2015)

\bibitem{scratch}
	Maloney, John H., et al.,
	Programming by choice: urban youth learning programming with scratch.
	ACM SIGCSE Bulletin,
	2008.

\bibitem{cb}
	CoolBasic - Game Creation \url{http://www.coolbasic.com/index.php?lang=fi} (luettu 30.1.2015)
	
\bibitem{vb.net}
	Visual Basic \url{https://msdn.microsoft.com/en-us/library/2x7h1hfk.aspx
} (luettu 30.1.2015)
	
% HTML&JS
\bibitem{mdn_canvas}
	Canvas - Web API Interfaces | MDN \url{https://developer.mozilla.org/en-US/docs/Web/API/Canvas_API} (luettu 10.11.2014)

\bibitem{mdn_about_js}
	About JavaScript - JavaScript | MDN \url{https://developer.mozilla.org/en-US/docs/Web/JavaScript/About_JavaScript} (luettu 10.11.2014)
	
\bibitem{w3c_html}
	HTML5 \url{http://www.w3.org/TR/html5/} (luettu 10.11.2014)

\bibitem{w3c_web_worker}
	Web Workers \url{http://www.w3.org/TR/workers/} (luettu 14.11.2014)

\bibitem{caniuse_canvas}
	Can I use... Support tables for HTML5, CSS3, etc \url{http://caniuse.com/#feat=canvas} (luettu 19.11.2014)

\bibitem{asm.js}
	asm.js \url{http://asmjs.org/spec/latest/} (luettu 23.1.2015)

% Uutiset
\bibitem{hs_kiuru}
	Ministeri Kiuru: Ohjelmointi peruskoulun opetussuunnitelmaan - Koulu - Kotimaa - Helsingin Sanomat \url{http://www.hs.fi/kotimaa/a1390279526604} (luettu 27.1.2015)
	
\bibitem{hs_eka}
	Ekaluokkalaisten ohjelmointiopetus alkaa vuonna 2016 – Espoon koodikerhosta saa esimakua - Tiede - Helsingin Sanomat \url{http://www.hs.fi/tiede/a1401942756096} (luettu 27.1.2015)

% Komponentit&kirjastot
\bibitem{ace_about}
	Ace - The Highg Performance Code Editor for the Web \url{http://ace.c9.io/} (luettu 11.11.2014)

\bibitem{requirejs}
	RequireJS \url{http://requirejs.org/} (luettu 21.1.2015)
	
\bibitem{r.js}
	jrburke/r.js \url{https://github.com/jrburke/r.js/} (luettu 21.1.2015)
	
\bibitem{jquery}
	jQuery \url{http://jquery.com/} (luettu 21.1.2015)

\bibitem{markdown}
	Daring Fireball: Markdown \url{http://daringfireball.net/projects/markdown/} (luettu 21.1.2015)
	
\bibitem{marked}
	chjj/marked \url{https://github.com/chjj/marked} (luettu 21.1.2015)
	
\bibitem{apache}
	Welcome! - The Apache HTTP Server Project \url{http://httpd.apache.org/} (luettu 1.2.2015)

\bibitem{django}
	The Web framework for perfectionists with deadlines | Django \url{https://www.djangoproject.com/} (luettu 1.2.2015)

\bibitem{postgresql}
	PostgreSQL: The world's most advanced open source database \url{http://www.postgresql.org/} (luettu 1.2.2015)

\end{thebibliography}
