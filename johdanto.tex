% Täällä lyhyesti tiivistetään raportin sisältö sekä kerrotaan sen taustoista
% Lisäksi täällä kerrotaan myös projektin syntyhistoriasta

\section{Johdanto}
\begin{comment}
Tietokoneet ymmärtävät konekieltä.
Konekieli muodostuu yksinkertaisisista komennoista,
joiden nopeaan suorittamiseen tietokoneet on suunniteltu.
Ihmiset ovat kuitenkin huonoja kirjoittamaan konekieltä
sen alkukantaisten komentojen vuoksi.
Ihmisiä varten onkin kehitelty erilaisia ohjelmointikieliä,
jotka sisältävät erilaisia abstraktioita ohjelmoinnin helpottamiseksi.

Muunnos ohjelmointikielestä konekieleen ei kuitenkaan aina ole yksinkertainen.
Monet ohjelmointikielten abstraktiot voidaan toteuttaa hyvinkin eri tavoin konekielessä,
sillä eri tietokonearkkitehtuurit sisältävät hyvinkin erilaisia konekielisiä komentoja.
Lisäksi lyhyistäkin ohjelmakoodinpätkistä voi syntyä pitkiä konekielisiä ohjelmia,
joten pientenkin ohjelmien muuttaminen konekieleksi voi viedä ohjelmoijalta tunteja.

Tämän prosessin automatisoimiseksi on kehitetty ohjelmia, joita kutsutaan kääntäjiksi.
Kääntäjä on yksinkertaisesti ohjelma, joka muuttaa koodia muodosta toiseen.
Usein tämä tarkoittaa ohjelmointikielen muuttamista konekieleksi,
mutta on myös olemassa kääntäjiä,
jotka kääntävät ohjelmointikieliä toisikseen.
Tällaisia kääntäjiä kutsutaan source-to-source-kääntäjiksi.
\end{comment}

Tässä raportissa kerron itse toteuttamastani
ohjelmointikielestä ja kääntäjästä -- EppaBasicista.
Samalla kerron myös kääntäjien toiminnasta yleisesti.

%Tässä raportissa kerron kääntäjien toiminnasta yleisesti
%sekä itse sunnittelemastani ja toteuttamastani %ohjelmointikielestä ja kääntäjästä -- EppaBasicista.

Suomessa on viimeaikoina puhuttanut ohjelmoinnin
opettaminen peruskouluissa.
Eräs ohjelmoinnin opetusta suunniteltaessa eteen tuleva
asia on ohjelmointikielen valinta.
Kielen tulee olla sellainen,
että sillä saa helposti jotain aikaan,
mutta sen tulee samalla olla tarpeeksi laaja,
jotta suurempienkin kokonaisuuksien tekeminen
oppimisen myötä olisi mielekästä.
Lisäksi kielen rakenteen on tuettava muiden,
yleisesti käytettyjen kielten hahmottamista.

Aikojen saatossa ohjelmoinnnin opettamiseen
on luotu monia eri kieliä ja menetelmiä.
Eräs ensimmäisistä juuri ohjelmoinnin opettamiseen
suunnitelluista kielistä oli BASIC
(Beginner's All-purpose Symbolic Instruction Code).
Ensimmäiset Basic-kielet olivat toiminnoiltaan rajoittuneita,
mutta pian kieleen tuli lisää ominaisuuksia.
Eräs tunnetuimmista Basic-kielistä on
Microsoftin kehittämä Visual Basic
(nykyisin Visual Basic .NET),
joka on edelleen laajasti käytössä.

Kehittämäni kieli EppaBasic on suunniteltu
erityisesti ohjelmoinnin aloittelijoille.
Kielellä on helppo lukea käyttäjän syötteitä
ja piirtää grafiikkaa,
joten yksinkertaisten pelien tekemisen oppii nopeasti.
Pelien tekemisen avulla aloittelijan into pysyy yllä,
sillä koko ajan saa nähdä työnsä tulokset.

EppaBasic on ollut onnistuneesti käytössä
Helsingin yliopiston järjestämällä ohjelmointileirillä,
jossa kielen avulla tutustuttiin erilaisten algoritmien toimintaan.
Graafisten ominaisuuksiensa ansiosta
EppaBasicilla voi helposti demonstroida
esimerkiksi järjestys- ja reitinhakualgoritmien toimintaa.
Kurssilta saatu palaute oli pääasiassa myönteistä
ja yli 80\% leiriläisistä suosittelisi kieltä toisille.
\fxnote{14/17 käyttäisi itse ja 15/17 suosittelisi}

Ohjelmointikielen toteuttamisessa on ollut itseni lisäksi mukana
Sami Kalliomäki (opiskelija, Karkkilan lukio),
joka on suunnitellut ja toteuttanut ohjelmointikielen ympärille tehdyn verkkosivuston
sekä auttanut muutenkin kielen suunnittelussa,
sekä Antti Laaksonen (Helsingin yliopisto),
joka on toiminut vaikuttavana taustavoimana projektin takana.
Lisäksi haluan kiittää kaikkia \#datatahti (IrcNet)- ja \#eppabasic (IrcNet)-kanavien keskustelijoita,
jotka ovat alun epäilyistä huolimatta tukeneet projektia.

\fxnote{Laajenna johtantoa: Muista basiceista ja niiden historiasta, koodaus peruskoulussa, eb ollut käytössä leireillä}

%Dart
%Codeacatemy
%
%Kopioi tänne githubista historiaa.