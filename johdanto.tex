% Täällä lyhyesti tiivistetään raportin sisältö sekä kerrotaan sen taustoista
% Lisäksi täällä kerrotaan myös projektin syntyhistoriasta

\section{Johdanto}
Suomessa on viime aikoina
puhuttanut ohjelmoinnin
opettaminen peruskouluissa
\cite{hs_kiuru}\cite{hs_eka}.
Ohjelmoinnin osaaminen
tukee matematiikan ja logiikan
oppimista, ja ohjelmoinnin
avulla voi nopeuttaa monia
toistoa vaativia töitä.
Myös tulevassa peruskoulun
opetussuunnitelmassa (OPS 2016)
linjataan, että peruskouluissa aletaan
vuonna 2016 opettaa ohjelmointia
\cite{OPS_2016}.
Ohjelmoinnista ei tule omaa oppiainetta,
mutta sitä on tarkoitus opettaa muiden
aineiden osana, aluksi erityisesti
matematiikan osana.

Yksi ohjelmoinnin opetusta suunniteltaessa eteen tuleva
asia on ohjelmointikielen valinta.
Opetussuunnitelmassa tähän ei
oteta kantaa.
Kielen tulisi olla sellainen,
että sillä saa helposti jotain aikaan,
mutta sen tulisi samalla olla tarpeeksi laaja,
jotta suurempienkin kokonaisuuksien tekeminen
taitojen karttuessa olisi mielekästä.
Lisäksi kielen rakenteen olisi tuettava muiden,
yleisesti käytettyjen kielten hahmottamista.

Ohjelmoinnin opettamiseen
on luotu monia eri kieliä ja menetelmiä
\cite{language_history}.
Yksi ensimmäisistä ohjelmoinnin opettamiseen
suunnitelluista kielistä oli BASIC
(Beginner's All-purpose Symbolic Instruction Code)
\cite{basic}.
Ensimmäiset Basic-kielet olivat toiminnoiltaan rajoittuneita,
mutta pian kieleen tuli lisää ominaisuuksia.
Eräs tunnetuimmista Basic-kielistä on
Microsoftin kehittämä Visual Basic
\cite{vb.net},
joka on edelleen laajasti käytössä.

Myös kehittämämme kieli EppaBasic
on Basic-pohjainen kieli.
EppaBasic on suunniteltu
erityisesti ohjelmoinnin aloittelijoille.
Kielellä on helppo lukea
näppäinten painalluksia
ja hiiren liikkeitä
sekä piirtää grafiikkaa,
joten esimerkiksi yksinkertaisten
pelien tekemisen oppii nopeasti.
Grafiikan ja pelien tekemisen avulla
aloittelijan into pysyy yllä,
sillä työnsä tulokset saa nähdä koko ajan.

EppaBasicia voi kokeilla osoitteessa
\url{http://eppabasic.fi}.
Kannattaa tutustua ainakin esimerkkikoodeihin
sekä ohjepaneelista löytyviin
ohjeiden esimerkkeihin.

EppaBasic on ollut onnistuneesti käytössä
Helsingin yliopiston LUMA-keskuksen elokuussa 2014
järjestämällä "Bittejä ja algoritmeja" -leirillä,
jossa kielen avulla peruskoulun
6--9-luokkalaiset
tutustuivat erilaisten algoritmien toimintaan.
Kurssilta saatu palaute oli myönteistä,
ja lähes kaikki leirille osallistuneet
suosittelisivat kieltä ystävilleen.
Kieltä on käytetty myös Karkkilan yhteiskoulussa
ohjelmoinnin perusteiden opettamiseen yläluokilla
lukuvuonna 2014--2015.

Projektia ovat olleet tekemässä Henrik Lievonen
(opiskelija, Helsingin matematiikkalukio) ja Sami Kalliomäki
(opiskelija, Karkkilan lukio). Henrik Lievonen
on kirjoittanut ohjelmointikielen kääntäjän ja
suoritusympäristön. Sami Kalliomäki on puolestaan
vastannut käyttöliittymän ja palvelinpuolen kehitystyöstä.
Tämän raportin ohjaajana toiminut
Antti Laaksonen (Helsingin yliopisto)
on toiminut myös vaikuttavana
taustavoimana EppaBasic-projektin takana.
Haluamme lisäksi kiittää kaikkia
\#datatahti- ja
\#eppabasic-IRC-kanavien keskustelijoita,
jotka ovat tukeneet projektia.