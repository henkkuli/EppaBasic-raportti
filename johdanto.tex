% Täällä lyhyesti tiivistetään raportin sisältö sekä kerrotaan sen taustoista
% Lisäksi täällä kerrotaan myös projektin syntyhistoriasta

\section{Johdanto}
Tietokoneet ymmärtävät konekieltä.
Konekieli muodostuu yksinkertaisisista komennoista,
joiden nopeaan suorittamiseen tietokoneet on suunniteltu.
Ihmiset ovat kuitenkin huonoja kirjoittamaan konekieltä
sen alkukantaisten komentojen vuoksi.
Ihmisiä varten onkin kehitelty erilaisia ohjelmointikieliä,
jotka sisältävät erilaisia abstraktioita ohjelmoinnin helpottamiseksi.

Muunnos ohjelmointikielestä konekieleen ei kuitenkaan aina ole yksinkertainen.
Monet ohjelmointikielten abstraktiot voidaan toteuttaa hyvinkin eri tavoin konekielessä,
sillä eri tietokonearkkitehtuurit sisältävät hyvinkin erilaisia konekielisiä komentoja.
Lisäksi lyhyistäkin ohjelmakoodinpätkistä voi syntyä pitkiä konekielisiä ohjelmia,
joten pientenkin ohjelmien muuttaminen konekieleksi voi viedä ohjelmoijalta tunteja.

Tämän prosessin automatisoimiseksi on kehitetty ohjelmia, joita kutsutaan kääntäjiksi.
Kääntäjä on yksinkertaisesti ohjelma, joka muuttaa koodia muodosta toiseen.
Usein tämä tarkoittaa ohjelmointikielen muuttamista konekieleksi,
mutta on myös olemassa kääntäjiä,
jotka kääntävät ohjelmointikieliä toisikseen.
Tällaisia kääntäjiä kutsutaan source-to-source-kääntäjiksi.

Tässä raportissa kerron kääntäjien toiminnasta yleisesti
sekä itse sunnittelemastani ja toteuttamastani ohjelmointikielestä ja kääntäjästä -- EppaBasicista.

Ohjelmointikielen toteuttamisessa on ollut itseni lisäksi mukana
Sami Kalliomäki (opiskelija, Karkkilan lukio),
joka on suunnitellut ja toteuttanut ohjelmointikielen ympärille tehdyn verkkosivuston
sekä auttanut muutenkin kielen suunnittelussa,
sekä Antti Laaksonen (\fxnote{arvonimi tähän}, Helsingin yliopisto),
joka on toiminut vaikuttavana taustavoimana projektin takana.
Lisäksi haluan kiittää kaikkia \#datatahti (IrcNet)- ja \#eppabasic (IrcNet)-kanavien projektiin osallistuineita henkilöitä.

%Dart
%Codeacatemy
%
%Kopioi tänne githubista historiaa.