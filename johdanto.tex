% Täällä lyhyesti tiivistetään raportin sisältö sekä kerrotaan sen taustoista
% Lisäksi täällä kerrotaan myös projektin syntyhistoriasta

\section{Johdanto}
Tässä raportissa kerron toteuttamastani
ohjelmointikielestä, EppaBasicista.
Esittelen EppaBasicin toinnallisuutta
sekä kerron, miten kieli sopii
ohjelmoinnin opettamiseen.
Lisäksi kerron, miten EppaBasic
kääntää ohjelman tietokoneen
ymmärtämään muotoon.

Ohjelmointikieli on tietokoneen ymmärrettäväksi
suunniteltu kieli. Toisin kuin ihmisten välisen kommunikaation
kielet, ohjelmointikielet ovat hyvin tarkkaan määriteltyjä ja
jäykkiä rakenteeltaan. Useimmat ohjelmointikielet ovat niin
sanotusti imperatiivisia, eli ne koostuvat tietokoneelle
annettavista käskyistä, jotka tietokone suorittaa järjestyksessä.
Erilaisilla hyppykomennoilla järjestystä voidaan muuttaa suorituksen
aikana ehtojen perusteella.

Suomessa on viime aikoina
puhuttanut ohjelmoinnin
opettaminen peruskouluissa
\cite{hs_kiuru}\cite{hs_eka}.
Ohjelmoinnin osaaminen
tukee matematiikan ja logiikan
oppimista, ja ohjelmoinnin
avulla voi nopeuttaa monia
toistoa vaativia töitä.
Myös tulevassa peruskoulun
opetussuunnitelmassa (OPS 2016)
linjataan, että peruskouluissa aletaan
vuonna 2016 opettaa ohjelmointia
\cite{OPS_2016}.
Ohjelmoinnista ei tule omaa oppiainetta,
mutta sitä on tarkoitus opettaa muiden
aineiden osana, aluksi erityisesti
matematiikan osana \cite{OPS_2016}\cite{hs_eka}.

Yksi ohjelmoinnin opetusta suunniteltaessa eteen tuleva
asia on ohjelmointikielen valinta.
Opetussuunnitelmassa tähän ei
oteta kantaa \cite{hs_eka}.
Kielen tulisi olla sellainen,
että sillä saa helposti jotain aikaan,
mutta sen tulisi samalla olla tarpeeksi laaja,
jotta suurempienkin kokonaisuuksien tekeminen
taitojen karttuessa olisi mielekästä.
Lisäksi kielen rakenteen olisi tuettava muiden,
yleisesti käytettyjen kielten hahmottamista.

Ohjelmoinnin opettamiseen
on luotu monia eri kieliä ja menetelmiä
\cite{language_history}.
Eräs ensimmäisistä ohjelmoinnin opettamiseen
suunnitelluista kielistä oli BASIC
(Beginner's All-purpose Symbolic Instruction Code)
\cite{basic}.
Ensimmäiset Basic-kielet olivat toiminnoiltaan rajoittuneita,
mutta pian kieleen tuli lisää ominaisuuksia.
Eräs tunnetuimmista Basic-kielistä on
Microsoftin kehittämä Visual Basic
(nykyisin Visual Basic .NET)
\cite{vb.net},
joka on edelleen laajasti käytössä.

Kehittämäni kieli EppaBasic
on myös
BASIC-pohjainen kieli.
EppaBasic on suunniteltu
erityisesti ohjelmoinnin aloittelijoille.
Kielellä on helppo lukea
näppäinten painalluksia
ja hiiren liikkeitä
sekä piirtää grafiikkaa,
joten esimerkiksi yksinkertaisten
pelien tekemisen oppii nopeasti.
Pelien tekemisen avulla aloittelijan into pysyy yllä,
sillä koko ajan saa nähdä työnsä tulokset.

EppaBasicia voi kokeilla osoitteessa
\url{http://eppabasic.fi}.
Kannattaa tutustua esimerkkikoodeihin
sekä ohjepaneelista löytyviin
ohjeiden esimerkkeihin.

EppaBasic on ollut onnistuneesti käytössä
Helsingin yliopiston elokuussa 2014
järjestämällä ohjelmointileirillä,
jossa kielen avulla peruskoulun
6--9-luokkalaiset
tutustuivat erilaisten algoritmien toimintaan.
Leirillä kielen graafisten ominaisuuksien
avulla havainnollistettiin erilaisten
algoritmien toimintaa.
Kurssilta saatu palaute oli pääasiassa myönteistä,
ja lähes kaikki leirille osallistuneet
suosittelisivat kieltä ystävilleen.

Ohjelmointiympäristön kehittämisessä on
ollut itseni lisäksi mukana
Sami Kalliomäki (opiskelija, Karkkilan lukio),
joka on suunnitellut ja toteuttanut
ohjelmointikielen ympärille tehdyn verkkosivuston
sekä auttanut muutenkin kielen suunnittelussa.
Tämän raportin ohjaajana toiminut
Antti Laaksonen (Helsingin yliopisto)
on toiminut myös vaikuttavana
taustavoimana EppaBasic-projektin takana.
Haluan kiittää heitä molempia
projektiin osallistumisesta.
Lisäksi haluan kiittää kaikkia
\#datatahti- ja
\#eppabasic-IRC-kanavien keskustelijoita,
jotka ovat tukeneet projektia.