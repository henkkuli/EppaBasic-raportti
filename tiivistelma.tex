% Täällä tiivistetään koko raportin sisältö

\section*{Tiivistelmä}
Viime aikoina mediassa on puhuttanut
ohjelmoinnin opetuksen aloittaminen
peruskouluissa.
Opetussuunnitelman mukaan vuonna
2016 kaikille peruskoulun aloittaville
aletaan opettaa ohjelmointia.
Valmistelutyö on kuitenkin vielä kesken,
ja esimerkiksi opetettavaa
ohjelmointikieltä ei ole vielä valittu.

EppaBasic on kehittämämme erityisesti
opetuskäyttöön suunnattu
ohjelmointikieli.
Kieli on helppo ottaa käyttöön
ja yksinkertainen oppia,
mutta samalla tarpeeksi laaja,
jotta pienien pelien ja ohjelmien
luominen olisi mielekästä.
EppaBasicin ohjelmointiympäristö ja kääntäjä
on toteutettu niin, että ne toimivat
kaikilla moderneilla selaimilla.

Tässä raportissa esittelemme EppaBasicin
ja kerromme sen tavoitteista ja toiminnasta.
EppaBasic on saatavilla osoitteessa \url{http://eppabasic.fi}.
EppaBasicin lähdekoodi on luettavissa
osoitteessa \url{https://github.com/scintillo/eppabasic}.

\vspace{50px}

Lately there has been discussion
in the media
about teaching programming in
Finnish comprehensive school.
According to the new curriculum,
every pupil entering the school
will be taught programming
from the year 2016 onwards.
Preparations are still in progress.
For example the language that
will be taught
has not been selected.

We have designed EppaBasic to be
specially used in teaching programming.
The language is easy to put to use
and simple to learn.
At the same time it is still
encompassing enough that
creating small games and programs
is meaningful.
The programming environment
and the compiler of EppaBasic
work in all modern browsers.

In this report we introduce EppaBasic
and tell about its objectives.
EppaBasic can be tested at
\url{http://eppabasic.fi}.
The source code is available at
\url{https://github.com/scintillo/eppabasic}.