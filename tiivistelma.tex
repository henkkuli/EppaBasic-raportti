% Täällä tiivistetään koko raportin sisältö

\section*{Tiivistelmä}
Viime aikoina mediassa on puhuttanut
ohjelmoinnin opetuksen aloittaminen
peruskouluissa.
Opetussuunnitelman mukaan vuonna
2016 kaikille peruskoulun aloittaville
aletaan opettaa ohjelmointia.
Valmistelutyö on kuitenkin vielä kesken,
ja esimerkiksi opetettavaa
ohjelmointikieltä ei ole vielä valittu.

EppaBasic on kehittämämme erityisesti
opetuskäyttöön suunnattu
ohjelmointikieli.
Kieli on helppo ottaa käyttöön
ja yksinkertainen oppia,
mutta samalla tarpeeksi laaja,
jotta pienien pelien ja ohjelmien
luominen olisi mielekästä.
EppaBasicin ohjelmointiympäristö ja kääntäjä
on toteutettu niin, että ne toimivat
kaikilla moderneilla selaimilla.

Tässä raportissa esittelemme EppaBasicin
ja kerromme sen tavoitteista ja toiminnasta.
EppaBasic on saatavilla osoitteessa \url{http://eppabasic.fi}.
EppaBasicin lähdekoodi on luettavissa
osoitteessa \url{https://github.com/scintillo/eppabasic}.

\vspace{50px}

\section*{Abstract}
Lately there has been discussion
in the media
about teaching programming in
Finnish schools.
According to the new curriculum,
from the year 2016 onwards,
every pupil starting school
will be taught programming.
However, preparations
are still in progress;
for example, the language to
be taught has not been selected.

We have especially designed EppaBasic to be
used in teaching programming.
The language is easy to put to use
and simple to learn.
At the same time, it is still
comprehensive enough to be
a meaningful tool for creating
small games and programs.
The programming environment
and the compiler of EppaBasic
have been put together in such
a way that they work in all
modern browsers.

In this report, we introduce EppaBasic
as well as discuss the objectives
and features of the language.
EppaBasic can be tested at
\url{http://eppabasic.fi}.
The source code is available at
\url{https://github.com/scintillo/eppabasic}.