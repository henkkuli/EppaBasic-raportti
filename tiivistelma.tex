% Täällä tiivistetään koko raportin sisältö

\section*{Tiivistelmä}
Suomessa ja maailmalla on viimeaikoina puhuttanut
ohjelmoinnin opettaminen jo peruskoulutasolla.
Olemme pyrkineet vastaamaan tähän tarpeeseen
kehittämällämme EppaBasic-ohjelmointikiellä, joka
on suunnattu opetuskäyttöön.

EppaBasicista tekevät erityisen sen helppo käyttöönotto
ja oppiminen. Kielellä saa myös helposti tuotettua
yksinkertaista grafiikkaa. Muilla ohjelmointikielillä
jo usein kehitysympäristön asentaminen muodostuu ongelmaksi.
Lisäksi oppija useimmiten joutuu kirjoittamaan pitkään
ainoastaan konsolipohjaisia ohjelmia, mikä useimmista
tuntuu vanhanaikaiselta. EppaBasicillä jo ensimäisissä
ohjelmissa pääse piirtämään näytölle.

Kielen kehitysympäristö, kääntäjä ja käännöstuote ovat
kaikki kirjoitettu JavaScriptillä, mikä tarkoittaa, että
ne toimivat kaikilla moderneilla selaimilla. Aloittamiseen
riittää pelkkä navigointi osoitteeseen \url{http://eppabasic.fi}.
Koodin voi myös helposti jakaa. Tästä esimerkkinä liitteen
Pong-peli löytyy osoitteesta \url{http://eppabasic.fi/#IHA6IE}.