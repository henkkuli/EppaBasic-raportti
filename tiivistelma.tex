% Täällä tiivistetään koko raportin sisältö

\section*{Tiivistelmä}
Viime aikoina mediassa on puhuttanut
ohjelmoinnin opetuksen aloittaminen
jo peruskouluissa.
Opetussuunnitelman mukaan vuonna
2016 kaikille peruskoulun aloittaville
aletaan opettaa ohjelmointia.
Valmistelutyö on kuitenkin vielä kesken,
ja esimerkiksi opetettavaa
ohjelmointikieltä ei ole vielä valittu.

EppaBasic on kehittämämme, erityisesti
opetuskäyttöön suunnattu
ohjelmointikieli.
Kieli on helppo ottaa käyttöön
ja yksinkertainen oppia,
mutta samalla tarpeeksi laaja,
jotta pienien pelien ja ohjelmien
luominen olisi mielekästä.
EppaBasicilla voikin luoda
helposti grafiikkaa.
EppaBasicin ohjelmointiympäristö ja kääntäjä
on toteutettu niin, että ne toimivat
kaikilla moderneilla selaimilla.

Tässä raportissa esittelemme EppaBasicin
ja kerromme sen tavoitteista ja toiminnasta.
EppaBasic on saatavilla osoitteessa \url{http://eppabasic.fi}.
EppaBasicin lähdekoodi on luettavissa
osoitteessa \url{https://github.com/scintillo/eppabasic}.