% Täällä kerrotaan aikaisemmista, vastaavista projekteista

\section{Aikaisempia järjestelmiä}
Monia kieliä on käännetty selainympäristöön
(esim. \eng{Python}, \eng{Lua}) \cite{repl.it}
Näillä kielillä grafiikan luominen on kuitenkin usein hankalaa.
Lisäksi useissa toteutuksissa ohjelmien on oltava lyhyitä,
sillä ohjelman suorittaminen jäädyttää selaimen,
kunnes ohjelma on suoritettu kokonaisuudessaan.
\fxnote{Esimerkkejä}

On myös kehitetty erilaisia ohjelmoinnin aloittelijoille suunnattuja,
selainympäristössä toimivia ohjelmointikieliä,
kuten esimerkiksi Massachusetts Institute of Technologyn
kehittämä Scratch-ohjelmointikieli.
\fxnote{Tarkista MIT ja Scratch}
Scratch on graafinen ohjelmointikieli,
mikä tarkoittaa,
että sen ohjelmoiminen tapahtuu raahamalla erilaisia komentopalikoita
peräkkäin ja sisäkkäin.
Vaikka tämä onkin nopeasti omaksuttava tapa tehdä omia ohjelmia,
suurien ohjelmien tekeminen on hankalaa,
samoin kuin siirtyminen myöhemmin muihiin kieliin.
Scratch on mielestämme liian rajoittunut kieli
yli kymmenvuotiaille nuorille.

\fxnote{Varmista sivu ja lisää linkki}
Codeacademy on tarjoaa omaa selaimessa toimivaa ohjelmointiympäristöään.
Kyseisessä ympäristössä ohjelmoidaan \eng{JavaScript}-kielellä,
joka muistuttaa läheisesti muita yleisesti käytettyjä kieliä,
kuten Java ja C++.
Ympäristössä syötteiden lukeminen ja piirtäminen
tapahtuu tapahtumien (eng. \eng{Events}) perusteella.
Mielestämme tapahtumien käyttäminen luo kuitenkin ohjelmakoodiin
aloittelijoille hankalasti ymmärrettävän suoritusjärjestyksen,
sillä koodin suoritus hyppii tapahtumien ohjelmarunkojen välillä.
\fxnote{Parempi sana}
Tämän takia oma ohjelmointikielemme rakenne ei pohjaudu eventteihin
vaan ohjelmakoodi suoritetaan aina ohjelmoijan kannalta loogisessa,
deterministisessä järjestyksessä.

\fxnote{Älä hauku! Basiceista, Pascalista, Logosta... muista aloituskielistä}
